\section{Method}
\subsection{Methodology and approach}
  \subsubsection{Alternatives}
  We found several alternatives for acquisition of memory from android devices.
  \begin{description}
    \item[dd on /dev/mem] \hfill \\
      This is a very simple method for acquiring memory. But /dev/mem can only be used directly when the kernel is 
      compiled with the STRICT\_DEVMEM flag off or with a kernel version pre 2.6. The first kernel version ever used on 
      android is version 2.6.25\footnote{http://elinux.org/Android\_Kernel\_Versions}. There is also a problem with 
      the dd application in android devices, it can not handle file offsets above 0x80000000 \cite{acq_vol_android_mem}.
    \item[fmem] \hfill \\
      fmem is a LKM (Loadable Kernel Module) which creates /dev/fmem. /dev/fmem is similar to /dev/mem, but without the limitations. 
      However, there are some issues using fmem on android devices running ARM including the problem with dd mentioned earlier \cite{acq_vol_android_mem}.
    \item[LiME LKM] \hfill \\
      LiME (Linux Memory Extractor) LKM\footnote{https://github.com/504ensicsLabs/LiME} provides a forensically sound method for acquiring memory from 
      memory \cite{heriyanto2013procedures}. LiME is formerly known as DMD (Droid Memory Dumpstr).
  \end{description}
  \subsubsection{Chosen approach}
  For acquisition of memory we have chosen to use the LiME LKM because of the problems described with the other methods.
\subsection{Environment}
When starting the project we needed to decide weather to use an physical device or an emulator
to conduct our experiment on.\\
To make the experiment as close to reality as possible it should have been done on a physical device, 
however, an emulator gives us a close match to reality and is easily replicable. Also, by using an emulator 
we avoid rooting our own phones, thus we avoid voiding warranty. Also, we found a guide on the Volatility 
wiki\footnote{https://github.com/volatilityfoundation/volatility/wiki/Android} for how to dump memory with 
an Android emulator. Therefore we chose to use an emulator.\\
Some deviation from the guide was done to be able conduct the experiment, due to the fact that we were running on 
a Linux system, not Mac OS X as the guide uses. This mostly involves editing the paths in the Makefiles you are editing 
to fit your system. % teit setning?
%A detailed guide on the setup can be found in the appendix \ref{setup}.\\

The emulator was set up on a computer running the latest Ubuntu (Ubuntu 14.04.1 LTS x86\_64).\\
The emulator we set up is based on the Nexus 7 (2012) with Android 4.2.2 Jelly Bean (API level 17) and The Linux Kernel 2.6.29. 
This was chosen because of convenience, since the guide we followed is using this and the 2.6 Kernel have very few restrictions.
\subsection{Tools}
What criteria we had for our tools
  \subsubsection{Volatility} % Det er jo en hel seksjon på dette? (av sindre)
  What is volatility? How could it be used?
  Support for many platforms: Windows, Linux, OS X
  Processes, network connections
  \subsubsection{PhotoRec}
  File recovery tool, supports many platforms and file systems, including memory images. 
  It Can extract images, text files and more.
  \subsubsection{Strings}
  Easy tool for getting all strings in a dump.
  \subsubsection{Hex editor}
  Useful for looking trough a dump.
\subsection{Experiments}
We did five dumps of the memory;
  \subsubsection{Clean dump}
  The first dump we did was right after boot, with factory settings and nothing done to the device 
  other than transferring the LiME LKM to the SD-card.
  \subsubsection{Pastebin entry}
  The second dump we did was after creating a pastebin entry on pastebin.com using the stock android browser (for 4.2.2),
  to see if we could find our entry in the memory.
  \subsubsection{Standard Text Message}
  The next dump was after sending a text message to the standard messages application. Since the emulator provides an telnet 
  interface, we where able to send a message using \texttt{sms send <phoneNumber> <textMessage>}.
  \subsubsection{Secure Text Message}
  Next we did the same as above, but after installing 
  TextSecure\footnote{https://play.google.com/store/apps/details?id=org.thoughtcrime.securesms}, which encrypts the messages. 
  \subsubsection{Screen lock}
  We also did a memory dump after creating a screen lock with a PIN code and using a passphrase.
