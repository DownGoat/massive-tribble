\section{Method}
\subsection{Methodology and approach}
  \subsubsection{Alternatives}
  We found several alternatives for acquisition of memory from android devices.
  \begin{description}
    \item[dd on /dev/mem] \hfill \\
      This is a very simple method for acquiring memory. But /dev/mem can only be used directly when the kernel is 
      compiled with the STRICT\_DEVMEM flag off or with a kernel version pre 2.6. The first kernel version ever used on 
      android is version 2.6.25 \footnote{http://elinux.org/Android\_Kernel\_Versions}.
    \item[fmem] \hfill \\
      fmem is a LKM (Loadable Kernel Module) which creates /dev/fmem. /dev/fmem is similar to /dev/mem, but without the limitations.
    \item[LiME LKM] \hfill \\
      LiME (Linux Memory Extractor) LKM \footnote{https://github.com/504ensicsLabs/LiME} provides a forensically sound method for acquiring memory from 
      memory \cite{heriyanto2013procedures}.
  \end{description}
  \subsubsection{Chosen approach}
  For acquisition of memory we have chosen to use the LiME LKM.
\subsection{Environment}
When starting the project we needed to decide weather to use an physical device or an emulator
to conduct our experiment on.\\
To make the experiment as close to reality as possible it should have been done on a physical device, 
however, an emulator gives us a close match to reality and is easily replicable. Also, by using an emulator 
we avoid rooting our own phones, thus we avoid voiding warranty.


\subsection{Tools}
What criteria we had for our tools
  \subsubsection{Volatility}
  What is volatility? How could it be used?
  Support for many platforms: Windows, Linux, OS X
  Processes, network connections
  \subsubsection{PhotoRec}
  What is Photorec? How does it differ from Volatility
\subsection{Experiments}
How we conducted experiments
  \subsubsection{Clean dump}
  What did we find?
  \subsubsection{Pastebin entry}
  Logs++
  \subsubsection{Text Message}
  Standard and secure
  \subsubsection{Screen lock}
  pass-phrase and pin code, identification
