\section{Further Work}
Through this project we have been looking at what can be found in cleartext in the memory of android phones to get an idea of how the volatile memory in android devices  can be useful. Based on our findings we can say that there are valuable information in the memory that can become relevant in a forensics case. From this there are a number of areas that can be researched.

The first and most critical area to make this valuable would be looking into how to actually get the memory of a live device, using the technique we have with the emulator, in most cases require a reboot of the device causing a lot of volatile memory to be lost.

Another topic for further work would be in developing tools to look through the memory data automatically, as in most cases you don't know exactly what you will be looking for some looking for some sort of pattern to find data consistently. This gets increasingly important as the memory in these devices is increasing quickly. Going for our test devices 800mb to newer devices 3gb of memory looking through this by hand is no longer realistic.

The fact that our research into secure applications isn't entirely conclusive, we couldn't find the encryption  key or any messages sent or received using this secure application, but they key should probably be in the memory somehow so how securely hashed or stored this is might be despite the fact its not in clear text isn't currently known. Using a different technique and a deeper search might reveal different results.  

Lastly, and this is a topic we intended to research further in our own paper is how long different sorts of information stays in memory. This is probably something that will be hard to test using an emulator as actual live devices will have a lot more apps and background services running. And you won't be likely to get as clean a memory as we did with our emulator tests. 