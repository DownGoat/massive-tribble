\section{Further Work}
Through this project we have been looking at what can be
found in cleartext in the memory of Android phones to get
an idea of how the volatile memory in Android devices can
be useful. Based on our findings we can say that there are
valuable information in the memory that can become relevant
in a forensics investigation. From this there are a number of
areas that can be researched.

The first and most critical area to make this valuable would
be looking into how to actually get the memory of a live
device, using the technique we have with the emulator, in most
cases require a ”cold” reboot of the device risking degrading
the volatile memory.

Another topic for further work would be in developing tools
to look through the memory data automatically, as in most
cases you do not know exactly what you will be looking for.
This would make a tool thats looking for patterns to find information useful.
This gets increasingly important as the memory capacity
in these devices is increasing quickly. The leap from our
emulated device with 800MB memory to newer devices with
2GB and higher capacity would make it not feasible to go
through the data manually.

The fact that our research into secure applications is not
entirely conclusive, we could not find the encryption key or any
messages sent or received using this secure application, but
they key is most likely in the memory. How securely hashed
or stored this is might be despite the fact its not in clear text
is not currently known. Using a different technique and a deeper
search might reveal different results.

Lastly, and this is a topic we intended to research further
in our own paper is how long different sorts of information
stays in memory. This is probably something that will be hard
to test using an emulator, as actual live devices will have a lot
more apps and background services running. And you will nott be
likely to get as clean a memory as we did with our experiments
using the emulator.

In this paper we have only looked Android running with
the Dalvik VM, in the more recent versions of
Android the experimental Android RunTime(ART) virtual
machine is introduced. This changes how apps are compiled
and run, from Dalviks just-in-time memory management to
ART’s ahead-of-time management. We believe this might
impact how memory is manage, and might cause different
results to the ones presented in this paper and might be a
topic for further research.

