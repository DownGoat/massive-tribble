\section{Further Work}
A topic for further work would be in developing tools
to analyze the memory. In most cases you do not know exactly what you will be looking for.
This gets increasingly important as the memory capacity
in these devices is increasing quickly. The leap from our
emulated device with 1024 MB memory to newer devices with
2 GB and higher capacity would make it not feasible to go
through the data manually.

The fact that our research into secure applications is not
entirely conclusive, we could not find the encryption key or any
messages sent or received using this secure application, but
they key is most likely in the memory. How securely hashed
or stored this is might be despite the fact it is not in clear text
is not currently known. Using a different technique and a deeper
search might reveal different results.

Lastly, and this is a topic we intended to research further
in our own paper, is how long different sorts of information
stays in memory. This is probably something that will be hard
to test using an emulator, as actual live devices will have a lot
more applications and background services running. And you will not be
likely to get as clean a memory as we did with our experiments
using the emulator.

In this paper we have only looked at Android running with
the Dalvik VM. In the more recent versions of
Android the experimental Android RunTime(ART) virtual
machine is introduced. This changes how applications are compiled
and run, from Dalviks \textit{just-in-time compilation} to
ART’s \textit{ahead-of-time compilation}. ART still uses the
\textit{dex} bytecode format that Dalvik uses for backward
compatibility, but it is very unlikely that any Dalvik specific
Volatility plugins will work with ART. To overcome this, plugins
needs to be able to read and understand where data is stored in the memory of
ART. When this work is done the application specific plugins will only need
minor modifications to work with the underlying code that reads the ART instead
of Dalvik.

