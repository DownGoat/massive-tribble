\section{Related Work}
In the last years there has been done a lot of research on acquisition of memory
and analysis techniques, targeting Linux and Android. The most common way to 
gain access to memory is gaining root privileges and load a new kernel into the 
phone. While this is not ideal, as it’s overwriting some of the memory in the 
progress, it is the only known way to get access to memory of all running processes.\\

In 2011 J. Sylve et al. presented a paper that described a forensic sound 
approach of acquiring Android memory. \cite{acq_vol_android_mem} This paper looked 
into ways of obtaining memory, with different tools and different approaches. The paper 
tries different ways to acquire memory and by using a feature(pmemsave) in the emulator. 
This creates a perfect snapshoot of the memory and could compare memory dumps from other 
tools and see how much they differed from the first. When compared to another method 
called Droid Memory Dumpstr (DMD) it was over 99\% identical. Since the memory dumps 
takes time, in our experiment it took us 5 minutes to dump 800mb of memory, some 
changes will naturally occur in the memory.\\

Based on these results, the paper presents this method for dumping memory is a 
forensic sound process and can be used as evidence in court. To analyze the memory 
acquired, different methods can be used, this paper also used Volatility.
One of the key benefits of Volatility is the ability to make your 
own plugins. The paper presents a new plugin that finds the regions where in memory 
each process is mapped, to make it easier to manually analyze it. Unfortunately we were not 
able to find this plugin to test it.\\
% Hvem er they? Var det ikke disse pluginsa vi testa?

This may be the first paper published that presents a method of accurate memory acquisition on
Android, as they write in their conclusions "To our knowledge, this is the first published work on
accurate physical memory acquisition and deep memory analysis of the Android kernel's structures".
