\section{Discussion}
In our project we where forced to discuss some issues for the project to be 
possible in our time frame. Our choice would need to be in a forensic sound manner 
so they could be applicable in court. 
\subsection{Real phone vs. emulator}
In a real case scenario, a forensic investigator would receive a physical device 
that the investigator should aquire evidence from. However in this project there 
are a number of reasons for why we didn't use a physical device.
\subsubsection{Warranty}
While many companies give instructions on how to unlock their bootloader, they also 
state that the warranty is void. However law-regulations in many European countries 
have better consumer laws that give the consumer rights if the flaw %mangel, norsk ord
is not related to rom or kernel of the device. Although we therefore would not have 
any real risk of loosing our rights to the producer we choose not to use our own 
devices since there always is some risk involved where the device could be bricked 
and consumer laws would not be applicable.
\subsubsection{Reproducibility}
When conducting scientific experiments, it is important to do something that is 
reproducible. By using a emulator we can document how it is set up so others can 
copy the environment and get the same results. Physical devices could be a issue 
where the device is out of production. There are also guides available where the 
environment and prerequisites are explained.
\subsection{Secure app - Textsecure}
One of our experiments was to look into how a app that gives the user higher 
privacy by encrypting its messages. Textsecure also enables encrypted point-to-
point encryption by use of PKI. As our results have shown we did not see the 
message in the memory dump like we did when tested against the stock sms app. 
However, it might be possible to extract the encryption keys where the database was 
aquired and decrypt the massages. Since textsecure is open source its possible to 
reverse engineer how it works %sindre, versågod :P
\subsection{Issues during the project}
Under the project we had some obstacles, we got a good start with researching and 
reading existing papers on the subject. However after we had decided on what 
direction we would take the project there where multiple issues when building the 
prerequisites for the kernel and emulator. There was some confusion from the guide 
on volatility page on what to follow. Since the guide mentioned OS X specifically 
we did not pay close enough attention to what we would have to deviate from it.
%Emulator, volatility, plugins, building kernel.
\subsection{Method}
As stated earlier in the report we had a number of methods we could approach 
obtaining memory from mobile devices. When choosing method we where looking for one 
that gave us the most forensic sound evidence. Originally our chosen method was 
using pmemsave %kommando, formatering
that was mentioned in a earlier paper\cite{acq_vol_android_mem}. However this did 
not work the emulator so the next best approach was chosen, where research sugests 
it gives 99.46\% identical memory dumps\cite{acq_vol_android_mem}.
\subsection{Applicable in forensic investigation}
When doing research on aquiring evidence it is important to see how it correlates 
to the real world. If the research is highly theoretical and not applicable it 
could be seen as useless. Here we discuss our thoughts on what would be different 
from a forensic investigation and our experiments.
\subsubsection{Kernel Module}
As explained in earlier chapters the LiME kernel module has to be compiled to the 
running kernel. The kernel also has to support loading of kernel modules. This is 
by default not allowed since it is a security vulnerability. This might be a 
problem if the device is running a stock rom, where it would be necessary to obtain 
the kernel source from the producer of the device. These are often not released for 
every device and could be a issue when getting evidence in the field. Since time is 
a issue when dealing with such volatile evidence it could be feasable to have 
precompiled kernels and modules that would work on most common devices. However 
this might be known by the criminals. %Finnes det noe research på mobiler brukt av 
kriminelle? statistikk :P