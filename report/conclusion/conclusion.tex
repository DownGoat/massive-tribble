\section{Conclusion}
<<<<<<< HEAD
Our project has proved that the volitile memory of android devices have valueable information for use 
in a forensic investigation.

%Time is a issue, precompiled\hardware atlernatives.
Each phone has its own ROM and kernel, therefore you would need kernels that support loading kernel 
modules. These should be precompiled since time is a issue when dealing with volatilie memory. An 
alternative would be to purchase hardware toolkits that can extract memory from the chip itself 
\footnote{Side med hw tool}

%anti-forensic; Secure apps. 5.0 Change to art, no plugin
Encryption and secure apps can make finding forensic investigation more difficult and we haven't found 
any research that has implemented countermesures against these anti-forensic tools. In newer versions 
of android the dalvik virtual machine has been replaced with android runtime(ART) where one of the new 
features have a smaller memory footprint. This is a issue in the forensic community where there is no 
plugin available that supports ART.

%reliable, luck? needle in the heystack
Our method for analysing the memory isnt nessaseraly realistic when used in forensics investigation, however it proves what might be available in the memory of android devices. To automate this process might make information from memory easier accessable and more reliable.

%malware (trojan horse defence)
Malware could have certan caracerictis that could be shown in memory, since mobile phones often stay 
powered on for longer periods they might be attractive targets for malware only running in memmory. 
This could possibly used for investigating cases with the trojan horse defense.
=======
Our project has proved that the volitile memory of android devices have valueable information for use in a forensic investigation.

Time is a issue, precompiled/hardware atlernatives.

anti-forensic

reliable, luck? needle in the heystack

malware (trojan horse defence)

>>>>>>> 37b25885e35a46bde07b0d8c25cce0696ba2adc8

\subsection{Real phone vs. emulator}
%Research applicable in real phone? Is a emulator environment relevant?
The major difference between a real phone and a emulator is when aquireing the memory dump is much 
easier. When using a real phone you run the risk of loosing live memory when trying to cooldboot and 
memory segments can be overwritten by the time a forensic analyst can take a dump.

The memory dump in it self is representable of a real device and give a clear indication of what 
information you can expect to find in the memory of android devices.

%\subsection{Method}
%Only one we got to work, 99.42\% the same.
%LiME kernel is commonly used by others.



