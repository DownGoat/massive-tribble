\section{Conclusion}
Our project has proved that the volatile memory of Android devices have valuable information for use 
in a forensic investigation. %Time is a issue, precompiled\hardware atlernatives.
Each phone has its own ROM and kernel, therefore you would need kernels that support loading kernel 
modules. These should be pre-compiled since time is a issue when dealing with volatile memory. An 
alternative would be to purchase hardware toolkits that can extract memory from the chip itself 
\footnote{Side med hw tool} % HUSK

%anti-forensic; Secure apps. 5.0 Change to art, no plugin
Encryption and secure applications can make finding forensic investigation more difficult and we have not found 
any research that has implemented countermeasures against these anti-forensic tools. In newer versions 
of Android the Dalvik VM has been replaced with Android Runtime(ART) where one of the new 
features give applications a smaller memory footprint. This is a issue in the forensic community where there is no 
plugin available that supports ART.

%reliable, luck? needle in the heystack
Our method for analyzing the memory is not necessarily applicable when used in forensic investigations, 
however it proves what might be available in the memory of Android devices. 
To automate this process might make information from memory easier accessible and more reliable.

%malware (trojan horse defence)
Malware could have certain characteristic that could be shown in memory, since mobile phones often stay 
powered on for longer periods they might be attractive targets for malware only running in memory. 
This could possibly be used for investigating cases with the trojan horse defence.

\subsection{Real phone vs. emulator}
%Research applicable in real phone? Is a emulator environment relevant?
The major difference between a physical phone and an emulator is that the memory dump is easier to acquire on an emulator. When using a real phone you run the risk of loosing live memory when trying to cooldboot and 
memory segments can be overwritten by the time a forensic analyst can take a memory dump.

The memory dump in it self is representable of a real device and give a clear indication of what 
information you can expect to find in the memory of Android devices.

%\subsection{Method}
%Only one we got to work, 99.42\% the same.
%LiME kernel is commonly used by others.



