\section{Results}
Here we will present our findings in the experiments conducted. All findings are
from analysis of memory dumps from each experiment.In this project we conducted 
a total of five experiments, where the purpose was to find forensic information 
from each dump.

\subsection{Clean dump}
The reason for this dump so we could see if we had set up our environment 
correctly and could read memory of the phone. The dump was also great for use 
when comparing to later dumps where we could see differences from a system with 
no significant use and compare them to the other experiments.

\subsubsection{Strings}

\subsubsection{Volatility}
With volatility and its plugins we where able to see the process list
\footnote{\url{https://github.com/volatilityfoundation/volatility/wiki/Linux\%20Command\%20Reference\#linux\_pslist}} 
and other use full information about the system.

\subsubsection{PhotoRec}
This program managed to get alot of files, such as .java code files and .txt 
files containing information from internal logfiles as syslog (dmesg). It also 
recovered some .png files with pictures used by the android launcher.

\subsection{Pastebin entry}
When starting to analyze this dump, we first started with the most basic. 
Strings and hex editor; by searching for "pastebin.com". Other findings in this 
process included a "timeline" on how the user got to the page patebin.com by 
examining the memory segments before the hit on our string.

\subsubsection{Hex editor}
This gave us a good way to examine the dump at its lowest level, by searching for 
"p.a.s.t.e.b.i.n...c.o.m" we found the url we posted in our emulator

\subsubsection{Strings}
Strings would show alot of results since there are often many hits on readable 
data in these kind of dumps, by piping the output to grep we where able to 
filter out text we wanted. Simply searching for pastebin.com/ we where able to 
find the page it was posted on. 
%Fant vi teksten som var skrevet også? brukte vi vol til dette? (yarascan)

\subsection{Standard Text Message}
This experiment was to see if it was possible to get recently received text 
messages from a live system for use in a forensic environment.

\subsubsection{Strings}
Same as above %?

\subsection{Secure Text Message}
In some cases the device might have used anti-forensic tools to hide their 
activity, we wanted to look into what information a memory analysis could 
retrieve.

\subsubsection{Dunno?}

\subsection{Screen lock}
If the device has a unlocked bootloader it would often be possible to use out 
method to retrieve memory of a device without wiping off all data from the 
device. This experiment was done to see if we could find a passphrase or 
pincode from the memory dump.

\subsubsection{Strings}
By searching for the pincode and passhrase we saw a pattern 
%mer info om dette, var noe tekst etter koden som var lik  elns.

