\section{Introduction}

%In computer forensics, memory/RAM have been common source of digital evidence. Modern smartphones also use memory for 
%running OS, application and user data which might be of value in forensics investigations. Especially when encryption 
%keys are required to decrypt critical digital evidence and when passwords for web and cloud services are needed to 
%access more sources of evidence. How to preserve digital evidence when analysing smartphone memory/RAM?

In digital forensics, memory is an important source of evidence. Most data from any type of device
is at some time stored in the memory.
Modern smartphones use memory of larger sizes like a conventional computer (laptop, desktop etc.).
These modern smarthpones contains memory and other user and application data which
might be of value in forensic investigations.
Data kept in the memory include geolocation, web browser history, SMS and encryption keys.
The last mentioned example (encryption keys), might be exclusively valuable as forensic evidence.
If a forensic investigator acquire these encryption keys, (s)he can get access to valuable/encrypted
information which the owner of the device have encrypted.

Acquiring memory may be quite simple in itself. But the real job is to analyze the data. Analyzing such
big data sets manually, is not feasible. To analyze the data faster and more accurately, certain tools
are used.

Something that can be a challenge when acquiring memory from a device is that the the memory is volatile.
In other words, the memory changes very frequently. As a matter of fact, it changes in nano
seconds[add source]. It is not possible to activate a write-blocker on memory as one would do when acquiring
evidence from for instance a hard drive. Therefore, as with all kinds of digital evidence, it is important to
gather the evidence as soon as possible. %or else it will be lost.
%How to preserve digital evidence when analyzing smartphone memory/RAM?
Different type of applications use different volumes of memory.

%devmem

%THe different approaches, OOV(order of volatility, ref to paper with games edit memory), reflections of forensic work in memory analysis)

%Expected output:
%One or more out of these:
%    - A framework for, or overview of, forensics methods and techniques (based 
%    on existing tools and methods) to conduct forensically sound analysis of 
%    smartphone memory
%
%    - A tool/toolkit for gathering digital evidence from smartphone memory
%
%    - Proposed method for how to extract smartphone memory as evidence data in 
%      a forensically sound manner.

