\section{Introduction}
In computer forensics, memory/RAM have been a common source of digital evidence. 
Modern smartphones use memory of larger sizes like a conventional computer (laptop, desktop etc.).
These modern smarthpones contains memory and other user and application  data which
might be of value in forensic investigations.
Data kept in the memory include geolocation, web browser history, SMS and encryption keys.
The last mentioned example (encryption keys), might be exclusively valuable as forensic evidence.
If a forensic investigator acquire these encryption keys, (s)he can get access to valuable/encrypted
information which the owner of the device have encrypted.

Something that can be a challenge when acquiring memory from a device is that the the memory is volatile.
In other words, the memory changes very frequently. 
How to preserve digital evidence when analyzing smartphone memory/RAM?

%Expected output:
%One or more out of these:
%    - A framework for, or overview of, forensics methods and techniques (based 
%    on existing tools and methods) to conduct forensically sound analysis of 
%    smartphone memory
%
%    - A tool/toolkit for gathering digital evidence from smartphone memory
%
%    - Proposed method for how to extract smartphone memory as evidence data in 
%      a forensically sound manner.

