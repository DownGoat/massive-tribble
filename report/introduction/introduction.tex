\section{Introduction}

%In computer forensics, memory/RAM have been common source of digital evidence. Modern smartphones also use memory for 
%running OS, application and user data which might be of value in forensics investigations. Especially when encryption 
%keys are required to decrypt critical digital evidence and when passwords for web and cloud services are needed to 
%access more sources of evidence. How to preserve digital evidence when analysing smartphone memory/RAM?

In digital forensics, memory is an important source of evidence. Most data from applications
is at some time stored in the memory of the device. Modern smartphones use memory of larger sizes like 
a conventional computer (laptop, desktop etc.). The memory in these devices contain user and 
application data which might be of value in forensic investigations. Data kept in the memory include 
information about running and terminated processes, open files and network information 
\cite{acq_vol_android_mem}. Other valuable information stored in memory can be geolocation, web-history 
and encryption keys. Encryption keys might be exclusively valuable as forensic evidence.
If the encryption keys are aquired during a forensic investigation the investigator could get access to 
valuabe information that might be encrypted.

There are several ways of acquiring memory from an android device. When having hardware access its possible to extract from the memory chips on the device, other methods by using software would in most cases require a unlocked bootloader and root access. As presented by \cite{acq_vol_android_mem}, one way to get root access is to use a privilege escalation exploit. %more about this

Acquiring memory may be quite simple in itself. But the other half of the job is to analyze the data.
Analyzing such big data sets manually is not feasible. To analyze the data faster and more accurately, certain tools could be used.

% Liker ikke "flyten" i dette avsnittet .P

To maintain evidence integrity, evidence acquisition should be performed automatically.

Something that can be a challenge when acquiring memory from a device is that the the memory is 
volatile. The memory is considered volatile since it requires power to be kept present \cite{the_art_of_mem}.
The memory changes very frequently. When looking at order of volatility, memory changes 
every 10ns[add source]. It is not possible to activate a write-blocker on memory as one would do when 
acquiring evidence from for instance a hard drive or a flash memory medium. Therefore, like when dealing with other sources of digital
evidence, it is important to gather the evidence as soon as possible. 

%How to preserve digital evidence when analyzing smartphone memory/RAM?
Different type of applications use different amounts of memory. Games use more memory than "regular" applications \cite{}.

Something that makes analyzing android memory a bit more difficult is that the memory addresses that a process
receives (including system processes), is fairly random because of address space layout randomization (ASLR). Therefore, looking for memory for a process in certain locations/addresses is a lost cause. %maybe citing here

%devmem

%THe different approaches, OOV(order of volatility, ref to paper with games edit memory), reflections of forensic work in memory analysis)

%Expected output:
%One or more out of these:
%    - A framework for, or overview of, forensics methods and techniques (based 
%    on existing tools and methods) to conduct forensically sound analysis of 
%    smartphone memory
%
%    - A tool/toolkit for gathering digital evidence from smartphone memory
%
%    - Proposed method for how to extract smartphone memory as evidence data in 
%      a forensically sound manner.

