\section{Introduction}

%In computer forensics, memory/RAM have been common source of digital evidence. Modern smartphones also use memory for 
%running OS, application and user data which might be of value in forensics investigations. Especially when encryption 
%keys are required to decrypt critical digital evidence and when passwords for web and cloud services are needed to 
%access more sources of evidence. How to preserve digital evidence when analysing smartphone memory/RAM?

In digital forensics, memory is an important source of evidence. Most data from any type of device
is at some time stored in the memory.
Modern smartphones use memory of larger sizes like a conventional computer (laptop, desktop etc.).
These modern smarthpones contains memory and other user and application data which
might be of value in forensic investigations.
Data kept in the memory include information about running and terminated processes, open files and network
information \cite{acq_vol_android_mem}. Other things stored in memory can be geolocation, web-history and
encryption keys. The last mentioned example (encryption keys), might be exclusively valuable as forensic evidence.
If a forensic investigator acquire these encryption keys, (s)he can get access to valuable/encrypted
information which the owner of the device have encrypted.

There are several ways of acquiring memory from an android device.
It is not possible to perform a memory dump from an android device right out of the box. To be able to do this,
a module has to be implemented. 
To be able to dump memory from an android device, a module has to be implemented.
As presented by \cite{acq_vol_android_mem}, one way to get root access is to use a
privilege escalation exploit. %more about this

Acquiring memory may be quite simple in itself. But the other half of the job is to analyze the data.
Analyzing such big data sets manually, is not feasible. To analyze the data faster and more accurately, certain tools
are used.

To maintain evidence integrity, evidence acquisition should be performed automatically.

Something that can be a challenge when acquiring memory from a device is that the the memory is volatile.
The memory is considered volatile since it requires power to be kept present \cite{the_art_of_mem}.
The memory changes very frequently. As a matter of fact, it changes in nano seconds[add source].
It is not possible to activate a write-blocker on memory as one would do when acquiring evidence from for
instance a hard drive. Therefore, as with other kinds of digital evidence, it is important to
gather the evidence as soon as possible. Something that can be done, is to put the device in hibernation or sleep mode.
When this is done, the device is put on pause. The memory stays untouched and can be acquired as is.
%How to preserve digital evidence when analyzing smartphone memory/RAM?
Different type of applications use different amounts of memory. Games use more memory than "regular" applications \cite{}.

Something that makes analyzing android memory a bit more difficult is that the memory addresses that a process
receives (including system processes), is fairly random. Therefore, looking for memory for a process in certain
locations/addresses is a lost cause. %maybe citing here

%devmem

%THe different approaches, OOV(order of volatility, ref to paper with games edit memory), reflections of forensic work in memory analysis)

%Expected output:
%One or more out of these:
%    - A framework for, or overview of, forensics methods and techniques (based 
%    on existing tools and methods) to conduct forensically sound analysis of 
%    smartphone memory
%
%    - A tool/toolkit for gathering digital evidence from smartphone memory
%
%    - Proposed method for how to extract smartphone memory as evidence data in 
%      a forensically sound manner.

