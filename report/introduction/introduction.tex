\section{Introduction}

In digital forensics, memory is an important source of evidence. Most of the data from applications is at some point 
stored in the memory of the device. Modern smartphones use memory sizes similar to conventional computers (laptop, 
desktop etc.). The memory in these devices contains user and application data which might be valuable in forensic 
investigations. Data kept in the memory include information about running and terminated processes, open files and 
network information \cite{acq_vol_android_mem}. Other valuable information stored in memory can be image files, GPS data, 
web-history and encryption keys. Encryption keys might be highly valuable as forensic evidence. If the encryption keys 
are acquired, an investigator could get access to valuable information that is encrypted. \\

The main ways of acquiring memory from Android devices is by installing kernel 
modules that allow you to extract the data. Typically this requires one to
have high user privileges (root) on the device, and often you will not have 
this on devices you have to analysis. To then be able to extract the memory you 
will need to escalate your privileges on the device, one way to achieve this is by
a privilege escalation exploit as presented by Case et al\cite{acq_vol_android_mem}.\\


Memory is highly volatile, acquiring it can therefore be a challenge. Memory is 
considered volatile for multiple reasons. It requires  power to be kept present
\cite{the_art_of_mem} and gets written to approximately every $10ns$ in regards
of order of volatility. %rar slutt på setning? nødvendig å nevne oov?
It is not possible to use a write-blocker on memory as you could when acquiring 
evidence from for instance a hard drive or a flash memory medium. Therefore, to 
the same degree as when dealing with other sources of digital evidence, it is 
important to gather the evidence as soon as possible. \\

%How to preserve digital evidence when analyzing smartphone memory/RAM?
Different type of applications use different amounts of memory. Games use more 
memory than applications that is not as graphical. Since games allocates more 
memory, also dynamically, objects not being used in the memory can be erased. 
This can result in loss of evidence that may be useful in an investigation.\\

\textit{Address Space Layout Randomization} (ASLR) makes analyzing the memory 
of processes more difficult, because the memory is not laid out in a linear 
fashion. This means that you have to leverage global data structures more for
locating where in memory data will be. ASLR will also allocate the data in 
different parts of the memory each time the same process is started. This means
that some piece of data, belonging to one process,  will not be in the same 
place in memory every time you start the process. ASLR was mainly developed for 
preventing buffer overflows \cite{prot_aslr}. But it also makes it more difficult
to analyzing memory.\\



%devmem

%THe different approaches, OOV(order of volatility, ref to paper with games edit memory), reflections of forensic work in memory analysis)

%Expected output:
%One or more out of these:
%    - A framework for, or overview of, forensics methods and techniques (based 
%    on existing tools and methods) to conduct forensically sound analysis of 
%    smartphone memory
%
%    - A tool/toolkit for gathering digital evidence from smartphone memory
%
%    - Proposed method for how to extract smartphone memory as evidence data in 
%      a forensically sound manner.

